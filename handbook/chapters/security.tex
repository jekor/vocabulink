\chapter{Security}

Many functions of Vocabulink require that we know the unique identity of the
client we're talking to. We identify the (authenticated) client with a number.
We could use the username, but because we're very permissive of the characters
in a username we want to transmit it as little as possible. A number is more
resilient.

To authenticate a member, we check their password against a hashed version
we've stored in our database. We store the passwords in hashed form so that if
our database is compromised we haven't given up our members' passwords (or at
least have made it more difficult).

Once a member has authenticated, we store their information in a cookie. We
then rely on the cookie, not information in our database, to authenticate the
member from then on. This allows us to protect the member more without
compromising their privacy. For example, we can add an IP address to the cookie
so that requests are harder to spoof, but we don't store the IP address in our
database so that if it's ever compromised the member's identity is somewhat
protected.
