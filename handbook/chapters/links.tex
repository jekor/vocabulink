\chapter{Links}

Links were originally made between pairs of lexemes. However, they are now more
free-form. A link can be made to a group of words (if there is no strict
equivalent) and to graphemes (such as a hiragana letter or Kanji).

Each link has a ``foreign'' and ``familiar'' side. Links are directed, and go
from the foreign to the familiar side. This is to reduce ambiguity, both for
the user and in the code. Reviews should always drill from foreign word to
familiar equivalent.

I use the term ``familiar'' instead of ``native'' since there's no reason to
limit yourself to native equivalents during language learning. If someone's
created a link to a word you've learned in a non-native language, why not make
use of it? The more associations, the better.

Even though link nodes may be graphemes, they cannot be arbitrary graphics.
They must be representable by a Unicode string.

\section{Search}

Search box access key: s